\subsection*{Overview}

Libmetal provides common user A\+P\+Is to access devices, handle device interrupts and request memory across the following operating environments\+:
\begin{DoxyItemize}
\item Linux user space (based on U\+IO and V\+F\+IO support in the kernel)
\item R\+T\+OS (with and without virtual memory)
\item Bare-\/metal environments
\end{DoxyItemize}

\subsection*{Project configuration}

The configuration phase begins when the user invokes C\+Make. C\+Make begins by processing the C\+Make\+Lists.\+txt file and the cmake directory. Some cmake options are available to help user to customize the libmetal to their own project.


\begin{DoxyItemize}
\item {\bfseries W\+I\+T\+H\+\_\+\+D\+OC} (default ON)\+: Build with documentation. Add -\/\+D\+W\+I\+T\+H\+\_\+\+D\+OC=O\+FF in cmake command line to disable.
\item {\bfseries W\+I\+T\+H\+\_\+\+E\+X\+A\+M\+P\+L\+ES} (default ON)\+: Build with application examples. Add -\/\+D\+W\+I\+T\+H\+\_\+\+D\+OC=O\+FF in cmake command line to disable the option.
\item {\bfseries W\+I\+T\+H\+\_\+\+T\+E\+S\+TS} (default ON)\+: Build with application tests. Add -\/\+D\+W\+I\+T\+H\+\_\+\+D\+OC=O\+FF in cmake command line to disable the option.
\item {\bfseries W\+I\+T\+H\+\_\+\+D\+E\+F\+A\+U\+L\+T\+\_\+\+L\+O\+G\+G\+ER} (default ON)\+: Build with default trace logger. Add -\/\+D\+W\+I\+T\+H\+\_\+\+D\+E\+F\+A\+U\+L\+T\+\_\+\+L\+O\+G\+G\+ER=O\+FF in cmake command line to disable the option.
\item {\bfseries W\+I\+T\+H\+\_\+\+S\+H\+A\+R\+E\+D\+\_\+\+L\+IB} (default ON)\+: Generate a shared library. Add -\/\+D\+W\+I\+T\+H\+\_\+\+S\+H\+A\+R\+E\+D\+\_\+\+L\+IB=O\+FF in cmake command line to disable the option.
\item {\bfseries W\+I\+T\+H\+\_\+\+S\+T\+A\+T\+I\+C\+\_\+\+L\+IB} (default ON)\+: Generate a static library. Add -\/\+D\+W\+I\+T\+H\+\_\+\+S\+T\+A\+T\+I\+C\+\_\+\+L\+IB=O\+FF in cmake command line to disable the option.
\item {\bfseries W\+I\+T\+H\+\_\+\+Z\+E\+P\+H\+YR} (default O\+FF)\+: Build for Zephyr environment. Add -\/\+D\+W\+I\+T\+H\+\_\+\+Z\+E\+P\+H\+YR=ON in cmake command line to enable the the option.
\end{DoxyItemize}

\subsection*{Build Steps}

\#\#\# Building for Linux Host 
\begin{DoxyCode}
$ git clone https://github.com/OpenAMP/libmetal.git
$ mkdir -p libmetal/<build directory>
$ cd libmetal/<build directory>
$ cmake ..
$ make VERBOSE=1 DESTDIR=<libmetal install location> install
\end{DoxyCode}


\subsubsection*{Cross Compiling for Linux Target}

Use \href{https://github.com/openamp/meta-openamp}{\tt meta-\/openamp} to build libmetal library. Use package {\ttfamily libmetal} in your yocto config file.

\subsubsection*{Building for Baremetal}

To build on baremetal, you will need to provide a toolchain file. Here is an example toolchain file\+: 
\begin{DoxyCode}
set (CMAKE\_SYSTEM\_PROCESSOR "arm"              CACHE STRING "")
set (MACHINE "zynqmp\_r5" CACHE STRING "")

set (CROSS\_PREFIX           "armr5-none-eabi-" CACHE STRING "")
set (CMAKE\_C\_FLAGS          "-mfloat-abi=soft -mcpu=cortex-r5 -Wall -Werror -Wextra \(\backslash\)
   -flto -Os -I/ws/xsdk/r5\_0\_bsp/psu\_cortexr5\_0/include" CACHE STRING "")

SET(CMAKE\_EXE\_LINKER\_FLAGS "$\{CMAKE\_EXE\_LINKER\_FLAGS\} -flto")
SET(CMAKE\_AR  "gcc-ar" CACHE STRING "")
SET(CMAKE\_C\_ARCHIVE\_CREATE "<CMAKE\_AR> qcs <TARGET> <LINK\_FLAGS> <OBJECTS>")
SET(CMAKE\_C\_ARCHIVE\_FINISH   true)

include (cross-generic-gcc)
\end{DoxyCode}

\begin{DoxyItemize}
\item Note\+: other toolchain files can be found in the {\ttfamily cmake/platforms/} directory.
\item Compile with your toolchain file. 
\begin{DoxyCode}
$ mkdir -p build-libmetal
$ cd build-libmetal
$ cmake <libmetal\_source> -DCMAKE\_TOOLCHAIN\_FILE=<toolchain\_file>
$ make VERBOSE=1 DESTDIR=<libmetal\_install> install
\end{DoxyCode}

\item Note\+: When building baremetal for Xilinx 2018.\+3 or earlier environments, add -\/\+D\+X\+I\+L\+I\+N\+X\+\_\+\+P\+R\+E\+\_\+\+V2019 to your C\+Make invocation. This will include the xilmem and xilstandalone libraries in your build. These libraries were removed in 2019.\+1.
\end{DoxyItemize}

\subsubsection*{Building for Zephyr}

The \href{https://github.com/zephyrproject-rtos/libmetal}{\tt zephyr-\/libmetal} implements the libmetal for the Zephyr project. It is mainly a fork of this repository, with some add-\/ons for integration in the Zephyr project.

Following instruction is only to be able to run test application on a Q\+E\+MU running a Zephyr environment.

As Zephyr uses C\+Make, we build libmetal library and test application as targets of Zephyr C\+Make project. Here is how to build libmetal for Zephyr\+: 
\begin{DoxyCode}
$ export ZEPHYR\_GCC\_VARIANT=zephyr
$ export ZEPHYR\_SDK\_INSTALL\_DIR=<where Zephyr SDK is installed>
$ source <git\_clone\_zephyr\_project\_source\_root>/zephyr-env.sh

$ cmake <libmetal\_source\_root> -DWITH\_ZEPHYR=on -DBOARD=qemu\_cortex\_m3 \(\backslash\)
  [-DWITH\_TESTS=on]
$ make VERBOSE=1 all
# If we have turned on tests with "-DWITH\_TESTS=on" when we run cmake,
# we launch libmetal test on Zephyr QEMU platform as follows:
$ make VERBOSE=1 run
\end{DoxyCode}


\subsection*{Interfaces}

The following subsections give an overview of interfaces provided by libmetal.

\subsubsection*{Platform and OS Independent Utilities}

These interfaces do not need to be ported across to new operating systems.

\paragraph*{I/O}

The libmetal I/O region abstraction provides access to memory mapped I/O and shared memory regions. This includes\+:
\begin{DoxyItemize}
\item primitives to read and write memory with ordering constraints, and
\item ability to translate between physical and virtual addressing on systems that support virtual memory.
\end{DoxyItemize}

\paragraph*{Log}

The libmetal logging interface is used to plug log messages generated by libmetal into application specific logging mechanisms (e.\+g. syslog). This also provides basic message prioritization and filtering mechanisms.

\paragraph*{List}

This is a simple doubly linked list implementation used internally within libmetal, and also available for application use.

\paragraph*{Other Utilities}

The following utilities are provided in \hyperlink{utilities_8h}{lib/utilities.\+h}\+:
\begin{DoxyItemize}
\item Min/max, round up/down, etc.
\item Bitmap operations
\item Helper to compute container structure pointers
\item ... and more ...
\end{DoxyItemize}

\paragraph*{Version}

The libmetal version interface allows user to get the version of the library. The version increment follows the set of rule proposed in \href{https://semver.org/}{\tt Semantic Versioning specification}.

\subsubsection*{Top Level Interfaces}

The users will need to call two top level interfaces to use libmetal A\+P\+Is\+:
\begin{DoxyItemize}
\item metal\+\_\+init -\/ initialize the libmetal resource
\item metal\+\_\+finish -\/ release libmetal resource
\end{DoxyItemize}

Each system needs to have their own implementation inside libmetal for these two A\+P\+Is to call\+:
\begin{DoxyItemize}
\item metal\+\_\+sys\+\_\+init
\item metal\+\_\+sys\+\_\+finish
\end{DoxyItemize}

For the current release, libmetal provides Linux userspace and bare-\/metal implementation for metal\+\_\+sys\+\_\+init and metal\+\_\+sys\+\_\+finish.

For Linux userspace, metal\+\_\+sys\+\_\+init sets up a table for available shared pages, checks whether U\+I\+O/\+V\+F\+IO drivers are avail, and starts interrupt handling thread. Please note that on Linux, to access device\textquotesingle{}s memory that is not page aligned, an offset has to be added to the pointer returned by mmap(). This {\ttfamily offset}, although it can be read from the device tree property exposed by the uio driver, is not handled yet by the library.

For bare-\/metal, metal\+\_\+sys\+\_\+init and metal\+\_\+sys\+\_\+finish just returns.

\subsubsection*{Atomics}

The libmetal atomic operations A\+PI is consistent with the C11/\+C++11 stdatomics interface. The stdatomics interface is commonly provided by recent toolchains including G\+CC and L\+L\+V\+M/\+Clang. When porting to a different toolchain, it may be necessary to provide an stdatomic compatible implementation if the toolchain does not already provide one.

\subsubsection*{Alloc}

libmetal provides memory allocation and release A\+P\+Is.

\subsubsection*{Locking}

libmetal provides the following locking A\+P\+Is.

\paragraph*{Mutex}

libmetal has a generic mutex implementation which is a busy wait. It is recommended to have OS specific implementation for mutex.

The Linux userspace mutex implementation uses futex to wait for the lock and wakeup a waiter.

\paragraph*{Condition Variable}

libmetal condition variable A\+P\+Is provide \char`\"{}wait\char`\"{} for user applications to wait on some condition to be met, and \char`\"{}signal\char`\"{} to indicate a particular even occurs.

\paragraph*{Spinlock}

libmetal spinlock A\+P\+Is provides busy waiting mechanism to acquire a lock.

\subsubsection*{Shmem}

libmetal has a generic static shared memory implementation. If your OS has a global shared memory allocation, you will need to port it for the OS.

The Linux userspace shmem implementation uses libhugetlbfs to support huge page sizes.

\subsubsection*{Bus and Device Abstraction}

libmetal has a static generic implementation. If your OS has a driver model implementation, you will need to port it for the OS.

The Linux userspace abstraction binds the devices to U\+IO or V\+F\+IO driver. The user applications specify which device to use, e.\+g. bus \char`\"{}platform\char`\"{} bus, device \char`\"{}f8000000.\+slcr\char`\"{}, and then the abstraction will check if platform U\+IO driver or platform V\+F\+IO driver is there. If platform V\+F\+IO driver exists, it will bind the device to the platform V\+F\+IO driver, otherwise, if U\+IO driver exists, it will bind the device to the platform U\+IO driver.

The V\+F\+IO support is not yet implemented.

\subsubsection*{Interrupt}

libmetal provides A\+P\+Is to register an interrupt, disable interrupts and restore interrupts.

The Linux userspace implementation will use a thread to call select() function to listen to the file descriptors of the devices to see if there is an interrupt triggered. If there is an interrupt triggered, it will call the interrupt handler registered by the user application.

\subsubsection*{Cache}

libmetal provides A\+P\+Is to flush and invalidate caches.

The cache A\+P\+Is for Linux userspace are empty functions for now as cache operations system calls are not available for all architectures.

\subsubsection*{D\+MA}

libmetal D\+MA A\+P\+Is provide D\+MA map and unmap implementation.

After calling D\+MA map, the D\+MA device will own the memory. After calling D\+MA unmap, the cpu will own the memory.

For Linux userspace, it only supports to use U\+IO device memory as D\+MA memory for this release.

\subsubsection*{Time}

libmetal time A\+P\+Is provide getting timestamp implementation.

\subsubsection*{Sleep}

libmetal sleep A\+P\+Is provide getting delay execution implementation.

\subsubsection*{Compiler}

This A\+PI is for compiler dependent functions. For this release, there is only a G\+CC implementation, and compiler specific code is limited to atomic operations.

\subsection*{How to contribute\+:}

As an open-\/source project, we welcome and encourage the community to submit patches directly to the project. As a contributor you should be familiar with common developer tools such as Git and C\+Make, and platforms such as Git\+Hub. Then following points should be rescpected to facilitate the review process.

\subsubsection*{Licencing}

Code is contributed to Open\+A\+MP under a number of licenses, but all code must be compatible with version the https\+://github.com/\+Open\+A\+M\+P/libmetal/blob/master/\+L\+I\+C\+E\+N\+S\+E.\+md \char`\"{}\+B\+S\+D License\char`\"{}, which is the license covering the Open\+A\+MP distribution as a whole. In practice, use the following tag instead of the full license text in the individual files\+: \begin{DoxyVerb}```
SPDX-License-Identifier:    BSD-3-Clause
```
\end{DoxyVerb}
 \subsubsection*{Signed-\/off-\/by}

Commit message must contain Signed-\/off-\/by\+: line and your email must match the change authorship information. Make sure your .gitconfig is set up correctly\+: \begin{DoxyVerb}```
git config --global user.name "first-name Last-Namer"
git config --global user.email "yourmail@company.com"
```
\end{DoxyVerb}
 \subsubsection*{gitlint}

Before you submit a pull request to the project, verify your commit messages meet the requirements. The check can be performed locally using the the gitlint command.

Run gitlint locally in your tree and branch where your patches have been committed\+: \begin{DoxyVerb}  ```gitlint```
\end{DoxyVerb}
 Note, gitlint only checks H\+E\+AD (the most recent commit), so you should run it after each commit, or use the --commits option to specify a commit range covering all the development patches to be submitted.

\subsubsection*{Code style}

In general, follow the Linux kernel coding style, with the following exceptions\+:


\begin{DoxyItemize}
\item Use /$\ast$$\ast$ $\ast$/ for doxygen comments that need to appear in the documentation.
\end{DoxyItemize}

The Linux kernel G\+P\+L-\/licensed tool checkpatch is used to check coding style conformity.\+Checkpatch is available in the scripts directory.

To check your $<$n$>$ commits in your git branch\+: 
\begin{DoxyCode}
./scripts/checkpatch.pl --strict  -g HEAD-<n>
\end{DoxyCode}
 \subsubsection*{Send a pull request}

We use standard github mechanism for pull request. Please refer to github documentation for help.

\subsection*{Communication and Collaboration}

\href{https://lists.openampproject.org/mailman3/lists/openamp-rp.lists.openampproject.org/}{\tt Subscribe} to the Open\+A\+MP mailing list(\href{mailto:openamp-rp@lists.openampproject.org}{\tt openamp-\/rp@lists.\+openampproject.\+org}).

For more details on the framework please refer to the the \href{https://github.com/OpenAMP/open-amp/wiki}{\tt Open\+A\+MP wiki}. 